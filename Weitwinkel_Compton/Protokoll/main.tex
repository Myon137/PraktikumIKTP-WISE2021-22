% Vorlage: https://www.pfsr.de/latex

% -- Anfang Präambel
\documentclass[german,  % Standardmäßig deutsche Eigenarten, englisch -> english
parskip=full,  % Absätze durch Leerzeile trennen
%bibliography=totoc,  % Literatur im Inhaltsverzeichnis (ist unüblich)
%draft,  % TODO: Entwurfsmodus -> entfernen für endgültige Version
]{scrartcl}

\usepackage[utf8]{inputenc}  % Kodierung der Datei
\usepackage[T1]{fontenc}  % Vollen Umfang der Schriftzeichen
\usepackage[ngerman]{babel}  % Sprache auf Deutsch (neue Rechtschreibung)

% Mathematik und Größen
\usepackage{amsmath}
\usepackage[locale=DE,  % deutsche Eigenarten, englisch -> US
separate-uncertainty,  % Unsicherheiten seperat angeben (mit ±)
]{siunitx}
\usepackage{physics}  % Erstellung von Gleichungen vereinfachen

\usepackage{graphicx}  % Bilder einbinden \includegraphics{Pfad/zur/Datei(ohne Dateiendung)}

% Gestaltung
\usepackage{booktabs}  % schönere Tabellen
\usepackage[toc]{multitoc}  % mehrspaltiges Inhaltsverzeichnis
\usepackage{csquotes}  % Anführungszeichen mit \enquote
\usepackage{caption}  % Anpassung der Bildunterschriften, Tabellenüberschriften
\usepackage{subcaption}  % Unterabbildungen, Untertabellen, …
\usepackage{enumitem}  % Listen anpassen
\setlist{itemsep=-10pt}  % Abstände zwischen Listenpunkten verringern

% Manipulation des Seitenstils
\usepackage[headtopline = .5pt]{scrlayer-scrpage}

% Bibliographie
\usepackage[backend=biber]{biblatex}
\addbibresource{bibliography.bib}

% SI-Einheiten darstellen
\usepackage{siunitx}

% Kopf-/Fußzeilen setzen
\pagestyle{scrheadings}  % Stil für die Seite setzen
\clearmainofpairofpagestyles  % Stil zurücksetzen, um ihn neu zu definieren
\automark{section}  % Abschnittsnamen als Seitenbeschriftung verwenden
\ofoot{\pagemark}  % Seitenzahl außen in Fußzeile
\ihead{\headmark}  % Seitenbeschriftung mittig in Kopfzeile

\usepackage[hidelinks]{hyperref}  % Links und weitere PDF-Features

% TODO: Titel und Autor, … festlegen
\newcommand*{\titel}{Weitwinkel-Compton-Koinzidenz-Kalibrierung}
\newcommand*{\autor}{Sebastian Thiede, Alexander Lettau}
\newcommand*{\abk}{WK}
\newcommand*{\betreuer}{V. Melzer}
\newcommand*{\messung}{04.11.2021 \& 11.11.2021}
\newcommand*{\ort}{ASB/406}

\hypersetup{pdfauthor={\autor}, pdftitle={\titel}}  % PDF-Metadaten setzen

% automatischen Titel konfigurieren
\titlehead{Praktikum des IKTP \abk \hfill TU Dresden}
\subject{Versuchsprotokoll}
\title{\titel}
\author{\autor}
\date{\begin{tabular}{ll}
Protokoll: & \today\\
Messung: & \messung\\
Ort: & \ort\\
Betreuer: & \betreuer\end{tabular}}

% -- Ende Präambel

\begin{document}
\begin{titlepage}
\maketitle  % Titel setzen
\tableofcontents  % Inhaltsverzeichnis setzen
\end{titlepage}

% ----- DOKUMENT ANFANG -----

\section{Einführung}
In diesem Praktikumsversuch soll ein Photonendetektor mittels Weitwinkel-Compton-Koinzidenz-Methode (WCKM) kalibriert werden.
Die WCKM ist eine Methode zur Energiekalibrierung von vor allem organischen Szintillatoren. Da organische Szintillatoren Atome mit niedrigen Kernladungszahlen (Niedrig-Z-Szintillatoren) verwenden kann keine Kalibrierung mittels Bestimmung des Vollenergiepeaks stattfinden, denn für die typischerweise verwendeten Kalibrierenergien (\SI{0.5}{\mega\electronvolt} bis \SI{1.5}{\mega\electronvolt}) überwiegt bei niedrigem Z die Compton-Streuung gegenüber dem Photoeffekt.
Für den Versuch wird außerdem ein HPGe-Detektor kalibriert um den Detektor auf Basis des organischen Szintillators mit diesem zu vergleichen. Der HPGe-Detektor kann aufgrund seiner höheren Kernladungszahl mittels Vollenergiepeakanalyse kalibriert werden.

Die zu bearbeitenden Aufgaben sind konkret:
\begin{enumerate}
    \item Vergleich der Detektorspektren
    \item Energiekalibrierung des HPGe-Detektors
    \item Untersuchung einzelner Streuwinkel
    \item Energiekalibrierung des organischen Szintillators
\end{enumerate}

\section{Theorie}

\subsection{Wechselwirkung von Photonen mit Materie}

Obwohl Photonen in vieler Weise mit Materie wechselwirken können sind für diesen Versuch nur zwei Prozesse von wesentlicher Bedeutung: Die Compton-Streuung und der Photoeffekt. In den folgenden Abschnitten werden beide näher erläutert.

\subsubsection{Photoeffekt}

Der Photoeffekt beschreibt

\subsubsection{Compton-Streuung}

\subsection{Weitwinkel-Compton-Koinzidenz-Methode}


\nocite{*} % alle resourcen auflisten
\printbibliography

% ----- DOKUMENT ENDE -----

\end{document}
